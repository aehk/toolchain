\documentclass[twoside]{article}

\usepackage{ucs}
\usepackage[utf8x]{inputenc}
\usepackage{graphicx}
\usepackage{wrapfig}
\usepackage{nopageno}
\usepackage{lipsum}

% ------
% Fonts and typesetting settings
\usepackage[sc]{mathpazo}
\usepackage[T1]{fontenc}
\linespread{1.05} % Palatino needs more space between lines
\usepackage{microtype}
\usepackage{ngerman}

% ------
% Page layout
\usepackage[hmarginratio=1:1,top=32mm,columnsep=20pt]{geometry}
\usepackage[font=it]{caption}
\usepackage{paralist}
\usepackage{multicol}

% ------
% Lettrines
\usepackage{lettrine}


% ------
% Abstract
\usepackage{abstract}
	\renewcommand{\abstractnamefont}{\normalfont\bfseries}
	\renewcommand{\abstracttextfont}{\normalfont\small\itshape}


% ------
% Titling (section/subsection)
\usepackage{titlesec}
\titleformat{\section}[block]{\large\scshape\centering{\thesection.}}{}{1em}{}
\titleformat{\subsection}[block]{\large\scshape\centering{\thesubsection}}{}{1em}{}


% ------
% Header/footer
\usepackage{fancyhdr}
	\pagestyle{fancy}
	\fancyhead{}
	\fancyfoot{}
	\fancyhead[C]{Einreichung zu TdSE 2015 $\bullet$ Industrievortrag $\bullet$ Jastram, Dorka}
	\fancyfoot[RO,LE]{\thepage}


% ------
% Clickable URLs (optional)
\usepackage{hyperref}

% PDF Metadata
\hypersetup{
    unicode=true,          % non-Latin characters in Acrobat’s bookmarks
    pdftoolbar=true,        % show Acrobat’s toolbar?
    pdfmenubar=true,        % show Acrobat’s menu?
    pdffitwindow=true,     % window fit to page when opened
    pdfstartview={FitH},    % fits the width of the page to the window
    pdftitle={Einreichung zur TdSE 2015: Solide Anforderungen dank ReqIF im europäischen Schienenverkehr},    % title
    pdfauthor={Michael Jastram, Moritz Dorka},     % author
    pdfsubject={Industrievortrag},   % subject of the document
    pdfnewwindow=true,      % links in new window
    colorlinks=true,       % false: boxed links; true: colored links
    linkcolor=blue,          % color of internal links
    citecolor=blue,        % color of links to bibliography
    filecolor=blue,      % color of file links
    urlcolor=blue           % color of external links
}

% ------
% Maketitle metadata
\title{\vspace{-15mm}%
	\fontsize{24pt}{10pt}\selectfont
	\textbf{Solide Anforderungen dank ReqIF im europäischen Schienenverkehr}
	}	
\author{%
	\large
	\textsc{Michael Jastram, Moritz Dorka} \\[2mm]
	\normalsize	Formal Mind GmbH \\
	\normalsize	TU Dresden
	\vspace{-5mm}
	}
\date{}

%\usepackage[left=2cm,top=0.9cm,right=2cm,bottom=3cm,nohead,nofoot]{geometry}
%\titleformat{\section}{\large\bfseries}{\thesection}{compact}{}


%%%%%%%%%%%%%%%%%%%%%%%%
\begin{document}

\maketitle
\thispagestyle{fancy}

\begin{abstract}
\noindent Im Zug von Düsseldorf über Brüssel nach Paris – das ist heute schon möglich, dank ETCS, dem European Train Control System.  Diese Spezifikation beschreibt die Kommunikation zwischen Signalanlagen und Zug. Leider wurde bei der Erstellung im Jahre 1998 auf den kleinsten gemeinsamen Nenner zurückgegriffen, MS Word. Das Ergebnis sind tausende Seiten, die schwer zu pflegen und inkonsistent formatiert und über viele Dateien verteilt sind.  Im Rahmen des ITEA-Projekts openETCS wurde ein Konverter entwickelt, mit dem diese Dokumente in das offene, international standardisierte ReqIF überführt werden kann.  Das ReqIF-basierte Anforderungsmodell wird dann im Projekt als Basis für die weitere Systemmodelllierung eingesetzt, wo es von verschiedenen Werkzeugen konsumiert werden kann.
\end{abstract}
	
\begin{multicols}{2}
\noindent 

%=========================================================================
\section{Einleitung}
% 1 Seite, zum Schluss schreiben
%=========================================================================

\lipsum[2]

%=========================================================================
\subsection{ETCS und openETCS}
% 1 Seite, Michael
%=========================================================================

Auch wenn es ETCS schon fast 20 Jahre gibt, so geht die Umsetzung in Europa doch eher träge voran \cite{traege_umsetzung}.

Status quo Zugverkehr – hier im Prinzip eine Zusammenfassung von Klaus-Rüdigers Folien und Veröffentlichungen.

Vision openETCS: Warum ETCS einen Anschub braucht

openETCS Projektstruktur von An

Open Proofs

Realität

%-------------------------------------------------------------------------
\subsection{Subset-026}
% ½ Seite, Moritz
%-------------------------------------------------------------------------

ETCS besteht aus diversen Teilsystemen, deren jeweilige Spezifikationen unabhängig voneinander gepflegt werden. Die Kernfunktionalität des Systems wird in dem gut 500-seitigen Dokument \glqq Subset-026\grqq\ beschrieben, das wiederum aus acht, in einzelnen Dateien verwalteten Kapiteln besteht.

Thematisch behandelt das \glqq Subset-026\grqq\ hauptsächlich einen Fahrzeugrechner, den sogenannten European Vital Computer (EVC) und dessen Schnittstellen. Der EVC kommuniziert über drei verschiedene Arten (punktförmiger Transponder, Leiterschleife und Funk) mit den streckenseitigen Einrichtungen und verfügt zudem über eine Mensch-Maschine-Schnittstelle (Driver Machine Interface) über die der Triebfahrzeugführer Eingaben vornehmen kann. All diese Daten verarbeitet der Rechner kontinuierlich und leitet daraus Aktionen für das Fahrzeug ab (Bremsen, Beschleunigen, Stromabnehmer heben/senken, \ldots ).

Neben dem \glqq{}Subset-026\grqq{} gibt es zahlreiche weitere Subsets, die sich mit den vom EVC angesteuerten Komponenten befassen. Deren breit gefächerte Inhalte reichen von funktionalen Anforderungen an eine fahrzeugseitige Aufzeichnungseinheit (eine Art \glqq Black-Box\grqq ) über streckenseitige Stellwerksschnittstellen bis hin zu kryptographischen Details für die Funkverbindung. Alle Spezifikationen werden von der European Railway Agency, einer EU-Behörde, zentral verwaltet und sind als Word- und PDF-Dokumente verfügbar.

%-------------------------------------------------------------------------
\subsection{ReqIF}
% ½ Seite, Michael
%-------------------------------------------------------------------------

%=========================================================================
\section{Von Word nach ReqIF}
%=========================================================================

Hier wird das eigentliche Arbeitsergebnis von Moritz vorgestellt.

%-------------------------------------------------------------------------
\subsection{Anforderungen im V-Modell}
% ½ Seite, Michael
%-------------------------------------------------------------------------

Im Schienenbereich bewegen wir uns im sicherheitskritischen Bereich, wo entsprechende Vorschriften herrschen.  Maßgeblich ist hier EN 50128 \cite{en50128}, eine Spezialisierung der EN 61508 für sicherheitsrelevante Software der Eisenbahn.  Hardware ist nicht im Scope von openETCS.  Der im Standard beschriebene Entwicklungsprozess orientiert sich am V-Modell.

Das V-Modell zeichnet sich unter anderem dadurch aus, dass Anforderungen und deren Verfolgbarkeit eine zentrale Rolle spielen: Anforderungen bilden den Ausgangspunkt der Entwicklung (\glqq{}links oben\grqq{}), wovon sich nachfolgend Artefakte bis zur Implementierung ableiten (\glqq{}Mitte unten\grqq{}). Wie viele Zwischenschritte es gibt, hängt von der Größe der Entwicklung ab.  Gleichzeitig gibt es eine Traceability zu den entsprechenden Tests auf den verschiedenen Ebenen (\glqq{}rechter Ast\grqq{}).



Zentrale Bedeutung von Anforderungen und Traceability / ReqIF passt strategisch sehr gut ins Projekt (offen, standardisiert, Eclipse-Implementierung besteht).
Siehe auch clause D.58 der EN 50128 wonach Traceability für SIL 3 und 4 \emph{mandatory} ist.

%-------------------------------------------------------------------------
\subsection{Word: Herausforderungen}
% ½ Seite, Moritz
%-------------------------------------------------------------------------

Microsoft Word ist ein Textverarbeitungsprogramm für die breite Masse. Es versucht dementsprechend möglichst viele Anwendungsbereiche abzudecken und dem Nutzer dabei möglichst wenige Einschränkungen zu machen. Dies führt unweigerlich zu einer sehr komplexen Software (\glqq eierlegende Wollmilchsau\grqq ) und spiegelt sich auch in dem zugrundeliegenden Dateiformat wider. Zudem verführt die Verfügbarkeit all dieser Funktionen den Nutzer bisweilen auch zu deren Verwendung -- was insbesondere im Umfeld von technischen Dokumentationen nicht unbedingt vorteilhaft sein muss.

Das größte Manko an Word aus Sicht des Anforderungsmanagements ist allerdings das Fehlen einer klaren semantischen Struktur. Es ist also nicht in jedem Fall ohne Weiteres möglich aus dem Fließtext eines solchen Dokuments einzelne, genau abgegrenzte Anforderungen zu extrahieren. Vielmehr sind hier domänenspezifische Algorithmen notwendig, die kontextabhängig sinnvolle, möglichst atomare Elemente in dem Datenstrom des Worddokuments erkennen und ihnen eindeutige Bezeichner für die spätere Traceability zuweisen können. Ebenso muss eine algorithmenbasierte Formalisierung der diversen Formatierungen erfolgen. Durch die Unterstützung von XHTML bietet ReqIF zwar die Voraussetzungen um zahlreiche einfachere Auszeichnungen (Fettdruck, Unterstreichungen, \ldots ) direkt zu übernehmen. Aber insbesondere bei stark layoutabhängigen Elementen (z.B. Pfeilen) hat sich die vorherige Umwandlung in eine Textform als sinnvoll erwiesen.

%-------------------------------------------------------------------------
\subsection{Lösungsansatz und Implementierung}
% 1 Seite, Moritz
%-------------------------------------------------------------------------

Hier schreiben wir noch nichts über die weitere Verwertung – das würde ich im Abschluss machen, und von hier darauf verweisen.

%-------------------------------------------------------------------------
\subsection{Einsatz in openETCS}
% ½ Seite, Moritz, Michael
%-------------------------------------------------------------------------

Einsatz der IDs in Reqtify, Nutzung von Eclipse RMF als Viewer für die Anforderungen. Verlinkung mit Papyrus über D\&D (Michael)

%=========================================================================
\section{Die Zukunft von openETCS}
%=========================================================================

Projekt endet dieses Jahr; Ergebnisse im Bereich Open Source eher enttäuschend. RMF die brauchbarste Komponente; wird durch den Word-Converter erheblich aufgewertet.

%-------------------------------------------------------------------------
\subsection{ReqIF in der Systementwicklung}
% 1 Seite, Michael
%-------------------------------------------------------------------------

Entwicklung von SysML2B, als alternative zu Scade.  Leichte Integration mit ReqIF möglich (Michael's Arbeit in Deploy)

Hinweis auf PolarSys, und entsprechende Aktivitäten

%-------------------------------------------------------------------------
\subsection{Verwertung}
% 1½ Seiten; Michael & Moritz
%-------------------------------------------------------------------------

openETCS-Tool auf gitHub

Word-Converter

formalmind Studio

sysml2b

%=========================================================================
\section{Fazit}
% ½ Seite
%=========================================================================


\end{multicols}

\titleformat{\section}[block]{\large\scshape\centering{}}{}{1em}{}

\bibliographystyle{amsalpha}
\bibliography{bibliography}


\end{document}

