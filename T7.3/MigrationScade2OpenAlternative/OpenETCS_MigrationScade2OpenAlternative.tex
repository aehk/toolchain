\documentclass{template/openetcs_report}
% Use the option "nocc" if the document is not licensed under Creative Commons
%\documentclass[nocc]{template/openetcs_article}
\usepackage{lipsum,url}
\usepackage{supertabular}
\usepackage{multirow}
\usepackage{url}
\usepackage{color, colortbl}
\definecolor{gray}{rgb}{0.8,0.8,0.8}
\usepackage[modulo]{lineno}
\graphicspath{{./template/}{.}{./images/}}

\begin{document}
\frontmatter
\project{openETCS}

%Please do not change anything above this line
%============================
% The document metadata is defined below

%assign a report number here
\reportnum{OETCS/WP7}

%define your workpackage here
\wp{Work-Package 7: ``Tool chain''}

%set a title here
\title{openETCS:\\ Migration from Scade to an Open Alternative}

%set a subtitle here
\subtitle{}

%set the date of the report here
\date{October 2015}

%document approval
%define the name and affiliation of the people involved in the documents approbation here
\creatorname{Silvano Dal Zilio}
\creatoraffil{LAAS-CNRS}

\techassessorname{}
\techassessoraffil{}

\qualityassessorname{}
\qualityassessoraffil{}

\approvalname{}
\approvalaffil{}

%define a list of authors and their affiliation here

\author{Silvano Dal Zilio}

\affiliation{LAAS-CNRS\\
  7 ave. Colonel Roche\\
  F-31004 Toulouse, France\\
  eMail: dalzilio@laas.fr \\
  WebSite: www.laas.fr}
  
%add yourself as author, if you contributed to the document

% define the coverart
\coverart[width=350pt]{openETCS_EUPL}

%define the type of report
\reporttype{Output Document}

%%%%%%%%%%%%%%%%%%%%%%%%%%%%%%%%%%%%%%%%%%%%%%%%%%%%%%%%%%%%%%%%%%%%%%

\begin{abstract}
  This document describes and evaluate a set of scenarios for
  migrating the openETCS OBU model, currently modeled using the
  proprietary Scade language, to an open format. After a brief
  description of the current modeling choices, we describe the
  different export formats supported by the Scade toolsuite. We base
  our possible migration scenarios on an analysis of the existing
  tools that can exploit these output formats.
\end{abstract}

%=============================
%Do not change the next three lines
\maketitle
\tableofcontents
\listoffiguresandtables
\newpage
%=============================

\chapter{Document Control}

\begin{tabular}{|p{4.4cm}|p{8.7cm}|}
\hline
\multicolumn{2}{|c|}{Document information} \\
\hline
Work Package &  WP7  \\
Deliverable ID & \\
\hline
Document title & openETCS Migration from Scade to an Open Alternative \\
Document version & 01 \\
Document authors (org.)  & \\
\hline
\end{tabular}

\begin{tabular}{|p{4.4cm}|p{8.7cm}|}
\hline
\multicolumn{2}{|c|}{Review information} \\
\hline
Last version reviewed & 01 \\
\hline
Main reviewers (org.) &  \\
\hline
\end{tabular}

\begin{tabular}{|p{2.2cm}|p{4cm}|p{4cm}|p{2cm}|}
\hline
\multicolumn{4}{|c|}{Approbation} \\
\hline
  &  Name & Role & Date   \\
\hline  
Written by    &  &   &  October 2015\\
\hline
Approved by & -- & -- & \\
\hline
\end{tabular}

\begin{tabular}{|p{2.2cm}|p{2cm}|p{3cm}|p{5cm}|}
\hline
\multicolumn{4}{|c|}{Document evolution} \\
\hline
Version &  Date & Author(s) & Justification  \\
\hline
00.0 &  &  & creation \\
\hline  
 & &  &   \\
\hline  
\end{tabular}
\newpage

% The actual document starts below this line
%=============================

\mainmatter

%%%%%%%%%%%%%%%%%%%%%%%%%%%%%%%%%%%%%%%%%%%%%%%%%%%%%%%%%%%%%%%%%%%%%%

\chapter{Introduction}

One of the main purpose of the openETCS project is to develop a model
of the ETCS using ``Open Standards'' on all levels. This includes
generating open hardware and software specifications, writing
interface definition using open languages, and developing open source
design tools. One of the most important output of the project is a
formal specification of the ETCS onboard unit (OBU) that follows the
specification defined by the European Railway Agency (ERA) in its
Subset-026 document~\cite{subset-026}. Nonetheless, for reasons that
we should briefly recall below, the proprietary language Scade---that
is a ``close-source'' solution---was chosen to develop the functional
model of the OBU.

Scade, now one of the product of ANSYS (\url{http://www.ansys.com/}),
is a formal modeling language and tool suite targeting safety-critical
embedded systems. It has been used for more than 15 years to develop a
broad range of control applications in the avionics, rail, and
automotive domains. The current Scade model for the OBU is a major
output of the project. The architectural model (written in SysML using
the Scade system tool) and the functional model (written in Scade
using the Scade suite tool) have been developed from scratch---without
the use of prior components---and cover all the aspects defined in the
ETCS specification. The models are made available on the GitHub of the
OpenETCS project
(\url{https://github.com/openETCS/modeling/tree/master/model}) and a
complete description is given in deliverable D3.5.3~\cite{d353}.

Several natural questions arise if we want to meet the ``open
philosophy'' of the OpenETCS project. For instance, would it be
possible to develop a model equivalent to what has been achieved with
Scade using an Open Source modeling language and Open Source tooling?
More interestingly, is it possible to capitalize on the substantial
modeling effort already invested in the Scade model and to migrate
this model into an open source format?  We concentrate on this last
question in this report.

Solving this migration problem raises several issues, for instance
concerning the simplicity of the migration process or the coherence of
the resulting models. Concerning the first point, \emph{simplicity},
we are of course interested by methods that are fully automatic and,
if possible, at least partially available (implemented). Concerning
the validity of the (target) models---and this is certainly the
critical stumbling block raised in this report---we need to evaluate
if all the hypotheses used to check the validity of the model (that is
to check how close the model is to the intent carried in the informal
specification of Subset-026) are safely ``carried-over'' during the
migration process. In the remainder of this text we use the term of
\emph{semantics preserving} migration to refer to this issue. This
issue is quite delicate. Indeed, Scade relies on a strong
implementation hypothesis, the so-called ``synchronous approach'', and
it is not clear how the properties of a model developed with a
synchronous approach can be preserved when transferred to another
setting, say for example the more ``hybrid approach'' favored by
languages such as Simulink. We give an example of the kind of
``semantic'' problem that can arise in the next section.

The rest of the report is as follows. In the next sections, we briefly
describe the Scade language and the particularity of the OBU Scade
model. We close the chapter with a short description of alternative
specification languages that could have been used for the model. In
chapter~\ref{cha:over-goal-transf}, we list the criteria that should
be taken into account when evaluating the possible migration
scenarios. Each migration scenario can be described by a
transformation from (one concrete syntax for) Scade to another
language. This is why, in chapter~\ref{cha:supp-outp-form}, we
describe the existing output formats supported by Scade suite. Before
concluding, in chapter~\ref{cha:possible-scenarios}, we define
possible scenarios based on the tools that can exploit these different
output formats and list the strengths and weaknesses of each approach.

% In the next sections we try to evaluate our scenarios with respects to
% these criteria and define a more precise list of criteria for what is
% a good migration scenario and evaluate our proposals based on these
% ones ....


% This is a central issue in our context. Indeed two objectives of the
% OBU models ; first to give a formal, unambiguous model of the
% Subset-026 in order to find inaccuracies or inconsistencies in the
% specification. A second objective is to define a functional model that
% is precise enough to generate code for prototyping.

% A more modest objective will be to produce an open source reference
% model that will not be further edited and that can serve as a basis
% for future iterations. While this second objective is more realistic,
% it is also less interesting with respect to the objectives of the
% project. In particular, it is difficult to reconcile this solution
% with another objective of the project, that is to provide a unique
% (standard) reference model in order for different implementations to
% be compatible with the hardware and software developed by concurrent
% providers (and compatible for several railway operators).

\section{Overview of Scade and Motivations Behind its Adoption in the
  Project}

XXXX WORK STILL IN PROGRESS XXXXX


Scade is a formal modeling language that provides both a graphical
(diagrammatic) and textual concrete syntax. Formal here means that the
intended meaning of a model can be defined unambiguously using a
mathematical framework. It is therefore possible to reason on models
and to prove properties on them with a very high-level of
confidence. More precisely, Scade models are synchronously clocked
state machines, interacting on infinite data streams, that can be
nested and intermixed with each other without limitations. Actually
the core of the Scade language is based on Lustre ...XXXX . Another
nice property is that Scade is an executable language with a
deterministic semantics, meaning that the ``execution'' of a Scade
model is deterministic XXXX????++++.  All these characteristics of
Scade (formal, concrete, executable and deterministic) made it
possible to develop DO-178B and EN50128 certified code generators to C
and ADA, which is not a small feat.

The language is supported by a toolbox, the Scade suite, that provides
an integrated design and development environment. Scade suite offers
several services: simulation and debugging at the model level; test
case execution; model test coverage measurement; code verification on
models; ... An extension of the tool. called Scade system, provides
additional functionalities that are of interest in the context of the
openETCS project.  Scade System provides systems modeling and model
generation based on the SysML standard and the Eclipse Papyrus open
source technology. In particular it offers gateway from and to SysML
and integration with Eclipse through the use of the Eclipse Modeling
Framework (EMF) and its integration with the Papyrus editor. Indeed,
Eclipse was selected as the de facto Open Source platform for
integrating the tools developed in the project while SysML was
selected for modeling the architecture of the system (and the SysML
diagrams have been edited using the Papyrus editor, with the
participation of CEA)


%%%%%%%%%%%%%%%%%%%%%%%%%%%%%%%%%%%%%%%%%%%%%%%%%%%%%%%%%%%%%%%%%%%%%%

Scade has been used, in production, on several big projects that
covers many application domains: avionics, rail, automotive, \dots It
allows the production of rapid prototype as well as of safety related
target system software. All these qualities are enough to motivate the
choice of Scade for modeling the OBU functions. 

But outside these purely factual advantages of Scade upon other
formats, there are also important ``contextual'' reasons. Most
importantly there does not exists an Open Source solution with the
same degree of maturity. For example, even though the Scade editor
requires a moderately high learning curve, it is a solid piece of
software that has been tested on many real-size examples. Another is
also , since Siemens, one of the major partner of the project, already
had a working knowledge of Scade and that some models related to the
ETCS (provided by Siemens) were already available. Henceforth,
considereing the deadlines of the project, Scade was the most (and the
less risky) solution for


We can also list several drawbacks to the choice of Scade. First the
choice of formalism means a ... this implies that there is a choice of
implementation imposed on the user of the models (namely the choice of
a synchronous architecture). This restrict the possible choices when
implementing. . The synchronous hypothesis is not XXXXX particularly
debilitating XXXX with operating on a single computer since it is not
very difficult to have all the components of the OBU to share a common
clok. Nonetheless this choice may have some non-trivial consequences
if a supplier decide to choose a more distributed architecture. Next
we can list the drawbacks related to the choice of a closed-source
solution:
%%
\begin{itemize}
\item risk of vendor lock-in, ... actually this is at the core of our migration scenario
  problem since
\item difficulty to have special needs taken into
  account or added to the tool (need of some critical mass in order
  to have on the ); 
\item risk on the perenity, this is experienced with the recent
  acquisition of Esterel TEchnologies by ANSYS (and Scade has known
  multiple ``owners'' since its debut)
\end{itemize}



\section{High-Level Description of the openETCS Scade Model}

The Scade model for the OBU is available on the GitHUb of the project
(\url{...}). The A description of the model is given in deliverable XXXX


XXX What were the decision on modeling; what are the choices in modeling
that are directly influenced by the choice of Scade ; what was very
useful to the modeler in the tooling XXX

For the purpose of this 
use of a large number of (complicated) types heavy use of state
machines, se for example the ManageLevelsAndModes
(\url{https://github.com/openETCS/modeling/tree/master/model/Scade/System/ObuFunctions/ManageLevelsAndModes/Modes})
use of iterators, that are not supported in other flavors of Lustre
and other comparable languages, e.g. Simulink

\section{Existing Open-Source Solutions Around (or Similar to)
  Scade}

link with the Lustre language and existing open-source tooling for Lustre, e.g. lus2lic or prelude, open-source lustre/Scade compiler ; readability of a pure Lustre model when compared to Scade ; licensing issue

possibility to choose another language that can embed Scade (not
necessarily synchronous), e.g. a subset of Simulink, for which there
is equivalent tooling available and better open-source support

issues concerning the certification of the tools (project requires EN50128)

Choice of a Domain Specific Modeling Language (DSML)

a posteriori evaluation of the use of SysML ; possibility to "reverse
engineer" a DSML from the existing Scade code Possible migration path
to one or several possible alternatives


SysML would have been a good choice and was explicitly. The drawback
is that necessity to choose a ``semantics'' (different tools have
different interpretations of the language, like for examples
transitions in State Diagram) and maturity of the tools. With the
knowledge accumulated during the modeling effort , but , taken into
account the temporal constraints of the project made it impossible to
have a backward-forward approach with co-developement. This initial
vision of the project for the future ?


%%%%%%%%%%%%%%%%%%%%%%%%%%%%%%%%%%%%%%%%%%%%%%%%%%%%%%%%%%%%%%%%%%%%%%

\chapter{Overall Goal of the Transformation (What do we Want)}
\label{cha:over-goal-transf}

We can try to establish a wish list:

loose as few information as possible (close to round-trip conversion)
solution is already existing or at least partially completed
available tooling for the format (editor, simulator, formal verification, ...)
integration in the OpenETCS toolsuite
SysML integration
is the translation semantics-preserving ? (use of a discrete time semantics)
traceability
certification (look at the output of WP2; any reference to a particular deliverable !?)
We do not expect to have a working implementation for the end of the project but we can define a roadmap of the possible scenarios for a following project.

Migration raises also several questions related to the life cycle of
the model, that is its ``evolvability'' in the future and its
maintainability. Indeed a model is never a finished object, it is
evolving and should be updated. The most ambitious goal is to
seamlessly replace the use of Scade suite by an open alternative. This
will be the most interesting solution. Nonetheless it is necessary to
acknowledge the fact that, exactly like software engineering,
\emph{model engineering} is essentially a social process and we cannot
expect that a transition will not disrupt the work of the modeling
team and the quality of its output.

%%%%%%%%%%%%%%%%%%%%%%%%%%%%%%%%%%%%%%%%%%%%%%%%%%%%%%%%%%%%%%%%%%%%%%

\chapter{Supported Output Formats from Scade}
\label{cha:supp-outp-form}

\section{XSCADE}

XML format keeping all editor-specific information (see Sect. 2 of
TechnicalManual\_SC-TM-R16).  Advantage: no information loss
Disadvantage: format not very well-documented (related XML Schema not
provided !?)

"parsers" are available from OSATE plugin and the S3 tool by Systerel (see below).

\section{SCADE}

SCADE Suite KCG can generate a \verb+kcg_xml_filter_out.scade+ file
which contains the translation of the graphical SCADE format into a
textual format Advantage: no information loss; except for the
placement of graphical elements if we want to target a graphical
format ; easier to parse Disadvantage: format not very well-documented

"parser" available in the Scade2B project (\url{https://github.com/cercles/scade2b})

\section{KCG}

textual format close to Lustre
Advantage: Many open-source Lustre implementation are available
Disadvantage: we loose the state machines and many information on the underlying "architecture" of the model (hierarchical components are flattened) ; name of variables and ports are quite obfuscated ; KCG is an extension of Lustre

Lustre is not a normalized language and the existing implementations have slightly different syntax. There are some works that have addressed the problem of converting "Scade's version" of Lustre to, for example, VERIMAG Lustre. Nonetheless none of these tools are mature.


\section{SysML}

SCADE System avoids duplication of efforts and inconsistencies between
system structural descriptions made of SysML Block Definition Diagram
(BDD) and Internal Block Diagram (IBD), and the full software
behavioral description designed through SCADE Suite models. Once the
system description is completed and checked, the individual software
blocks in the system can be refined in the form of models in SCADE
Suite or in the form of manually developed source code. Automatic and
DO-178B Level A-qualified code generation can then be applied to the
SCADE Suite models. Moreover, the SCADE System description can be used
as the basis to develop scripts that will automatically integrate the
complete application software.


failed attempt of using Scade System during the work on the OpenETCS
Data Dictionnary that lists the different values and datagrams used in
the ETCS specification together with their datatypes.


%%%%%%%%%%%%%%%%%%%%%%%%%%%%%%%%%%%%%%%%%%%%%%%%%%%%%%%%%%%%%%%%%%%%%%

\chapter{Possible Scenarios}
\label{cha:possible-scenarios}

\section{Conversion to the AADL}

See \url{http://www.aadl.info/aadl/currentsite/}
The OSATE 2 Eclipse plugin provides an importer from XSCADE to AADL
Each component in the diagram is mapped into an AADL system component
SCADE component connections is mapped using AADL event data port connections
SCADE state machines are translated into AADL behavior state machine
we need to test the transformation on the current Scade model

\section{Conversion to HLL}

Intermediate format of Systerel S3 tool

the tool is based on a XSCADE parser ; need to extend HLL or adapt the
transformation to better take into account (nested) state machines HLL
has already been used succesfully on the OpenETCS Scade model
(\url{https://raw.githubusercontent.com/openETCS/validation/master/VnVUserStories/VnVUserStorySysterel/05-Work/S3/Proof_SoM/Modes.hll})

\section{Conversion to SysML}

with added behavioral information in Lustre

Scade Suite provides an extension mechanism based on the use of TCL scripts.
Conversion to Simulink (!?)


%%%%%%%%%%%%%%%%%%%%%%%%%%%%%%%%%%%%%%%%%%%%%%%%%%%%%%%%%%%%%%%%%%%%%%

\chapter{Conclusion}


Solving this migration problem raises several issues, for instance
concerning the simplicity of the migration process or the coherence of
the resulting models. Concerning the first point, we are of course
interested by methods that are fully automatic and, if possible, at
least partially available (implemented). It should not come as a
surprise that there is no existing solution that satisfies all of our
needs. Two main reasons may be identified to explain this situation,
that goes beyond the simplistic argument of vendor lock-in: first,
there is a relatively small user-base for Scade (and therefore less
incentive for third-parties to develop export tools); secondly, there
is a lack of compelling scenarios fro trying to export Scade models to
another format, except for code generation. Indeed, Scade targets
mainly the detailed design phase (the latter design step of the
traditional development cycle) and therefore Scade models have no use
after the code generation or model verification activities, which are
both supposed to be performed using the Scade tools.  Nonetheless we
provide some plausible scenarios for ``escaping'' the Scade as
uncovered some existing state of the art that gives interesting (and
plausible) scenarios.


%%%%%%%%%%%%%%%%%%%%%%%%%%%%%%%%%%%%%%%%%%%%%%%%%%%%%%%%%%%%%%%%%%%%%%

\bibliographystyle{unsrt}
\bibliography{ref}


%===================================================
%Do NOT change anything below this line

\end{document}


%%% Local Variables:
%%% mode: latex
%%% TeX-master: t
%%% End:
