\documentclass{template/openetcs_report}
% Use the option "nocc" if the document is not licensed under Creative Commons
%\documentclass[nocc]{template/openetcs_article}
\usepackage{lipsum,url}
\usepackage{supertabular}
\usepackage{multirow}
\usepackage{url}
\usepackage{color, colortbl}
\definecolor{gray}{rgb}{0.8,0.8,0.8}
\usepackage[modulo]{lineno}
\graphicspath{{./template/}{.}{./images/}}

\begin{document}
\frontmatter
\project{openETCS}

%Please do not change anything above this line
%============================
% The document metadata is defined below

%assign a report number here
\reportnum{OETCS/WP7}

%define your workpackage here
\wp{Work-Package 7: ``Tool chain''}

%set a title here
\title{openETCS:\\ Migration from Scade to an Open Alternative}

%set a subtitle here
\subtitle{}

%set the date of the report here
\date{October 2015}

%document approval
%define the name and affiliation of the people involved in the documents approbation here
\creatorname{Silvano Dal Zilio}
\creatoraffil{LAAS-CNRS}

\techassessorname{}
\techassessoraffil{}

\qualityassessorname{}
\qualityassessoraffil{}

\approvalname{}
\approvalaffil{}

%define a list of authors and their affiliation here

\author{Silvano Dal Zilio}

\affiliation{LAAS-CNRS\\
  7 ave. Colonel Roche\\
  F-31004 Toulouse, France\\
  eMail: dalzilio@laas.fr \\
  WebSite: www.laas.fr}

\author{Marc Behrens}
\affiliation{DLR}

\author{Raphaël Faudou}
\affiliation{SAMARES Engineering}

\author{David Mentré}
\affiliation{MERCE}

\author{Xavier Zeitoun}
\affiliation{CEA}

  
%add yourself as author, if you contributed to the document

% define the coverart
\coverart[width=350pt]{openETCS_EUPL}

%define the type of report
\reporttype{Output Document}

%%%%%%%%%%%%%%%%%%%%%%%%%%%%%%%%%%%%%%%%%%%%%%%%%%%%%%%%%%%%%%%%%%%%%%

\begin{abstract}
  This document evaluates possible scenarios for migrating the
  openETCS OBU model, currently developed using the proprietary Scade
  language, to an open format. After a brief description of the
  current modeling choices, we describe the different export formats
  supported by the Scade toolsuite. We base our possible migration
  scenarios on an analysis of the existing tools that can exploit
  these output formats.
\end{abstract}

%=============================
%Do not change the next three lines
\maketitle
\tableofcontents
%\listoffiguresandtables
\newpage
%=============================

\chapter{Document Control}

\begin{tabular}{|p{4.4cm}|p{8.7cm}|}
\hline
\multicolumn{2}{|c|}{Document information} \\
\hline
Work Package &  WP7  \\
Deliverable ID & \\
\hline
Document title & openETCS Migration from Scade to an Open Alternative \\
Document version & 1.1 \\
Document authors (org.)  & 

{Silvano Dal Zilio} (LAAS-CNRS)\\
\hline
\end{tabular}

\begin{tabular}{|p{4.4cm}|p{8.7cm}|}
\hline
\multicolumn{2}{|c|}{Review information} \\
\hline
Last version reviewed & 01 \\
\hline
Main reviewers (org.) &  {Marc Behrens} (DLR); Raphaël Faudou
                           (SAMARES); David Mentré (MERCE); Xavier
                           Zeitoun (CEA) \\
\hline
\end{tabular}

\begin{tabular}{|p{2.2cm}|p{4cm}|p{4cm}|p{2cm}|}
\hline
\multicolumn{4}{|c|}{Approbation} \\
\hline
  &  Name & Role & Date   \\
\hline  
Written by    &  &   &  October 2015\\
\hline
Approved by & -- & -- & \\
\hline
\end{tabular}

\begin{tabular}{|p{2.2cm}|p{2cm}|p{3cm}|p{5cm}|}
\hline
\multicolumn{4}{|c|}{Document evolution} \\
\hline
Version &  Date & Author(s) & Justification  \\
\hline
01.0 &  October 2015 & S. Dal Zilio  & creation \\
\hline  
01.1 &  November 2015 & S. Dal Zilio  & added comments from reviewers \\
\hline  
 & &  &   \\
\hline  
\end{tabular}
\newpage

% The actual document starts below this line
%=============================

\mainmatter

%%%%%%%%%%%%%%%%%%%%%%%%%%%%%%%%%%%%%%%%%%%%%%%%%%%%%%%%%%%%%%%%%%%%%%

\chapter{Introduction}
\label{cha:introduction}

One of the main purpose of the openETCS project is to develop a model
of the ETCS on-board unit (OBU) using ``Open Standards'' at every
stages. This includes generating open hardware and software
specifications, writing interface definition using open languages, and
developing open source design tools for supporting this effort. One of
the most important output of the project is a formal specification of
the ETCS that follows the specification defined by the European
Railway Agency (ERA) in its Subset-026
document~\cite{subset-026}. Nonetheless, for reasons that we will
briefly recall in this chapter, the proprietary language Scade---that
is a ``closed-source'' solution---was chosen to develop the functional
model of the OBU.

Scade is a formal modeling language and tool suite targeting
safety-critical embedded systems. It is developed by Esterel
Technologies, now a company of the ANSYS group
(\url{http://www.ansys.com/}). Scade has been used for more than 15
years to develop a broad range of control applications in the
avionics, rail, and automotive domains. The current Scade model for
the OBU is a major output of the project. The architectural model
(modeled in SysML using the Scade system tool) and the functional
model (modeled in Scade using the Scade suite tool) have been
developed from scratch---without the use of prior components---and
cover all the aspects defined in chapters 3, 4,5, 7 and 8 of the
\emph{ETCS specification}\footnote{In the remainder of theis text we
  will sometimes use ``ETCS specification'' instead of the longer form
  ``SRS subset 26 specification''.}. The models are made available on
the GitHub of the openETCS project
(\url{https://github.com/openETCS/modeling/tree/master/model}) and a
complete description is given in deliverable D3.5.3~\cite{d353}.

Several natural questions arise if we want to meet the ``open
philosophy'' of the openETCS project. For instance, would it be
possible to develop a model equivalent to what has been achieved with
Scade using an open source modeling language and open source tooling?
More interestingly, is it possible to capitalize on the substantial
modeling effort already invested in the Scade model and to migrate
this model into an open source format?  We concentrate on this last
question in this report.

Solving this migration problem raises several issues, for instance
concerning the simplicity of the migration process or the coherence of
the resulting models. Concerning the first point, \emph{simplicity},
we are of course interested by methods that are fully automatic and,
if possible, at least partially available (implemented). This is why
we will concentrate on solutions based on a source-level conversion
from Scade. Concerning the validity of the (target) models---and this
is certainly the critical stumbling block raised in this report---we
need to evaluate if all the hypotheses used to check the validity of
the model (that is to check how close the model is to the informal
specification of Subset-026) are safely ``carried-over'' during the
migration process. In the remainder of this text we use the term of
\emph{semantics preserving} migration to refer to this issue. This
issue is quite delicate. Indeed, Scade relies on a strong
implementation hypothesis, the so-called ``synchronous approach'', and
it is not obvious how the properties of a model developed with a
synchronous approach can be preserved when transferred to another
setting, say for example the more ``hybrid approach'' favored by
languages such as Simulink. To state this problem simply: we may
always translate a piece of text from one language to another, but it
is difficult to prove that the original meaning is preserved in the
process. We sketch an example (see
Sect.~\ref{sec:associated-drawbacks}) of the kind of ``semantic''
problem that can arise in the following sections.

The rest of the report is as follows. In the remaining sections of
chapter~\ref{cha:introduction}, we briefly describe the Scade language
and the particularity of the OBU Scade model. In
chapter~\ref{cha:over-goal-transf}, we list the criteria that should
be taken into account when evaluating the possible migration
scenarios. Each migration scenario can be described by a
transformation from (one concrete syntax for) Scade to another
language. This is why, in chapter~\ref{cha:supp-outp-form}, we
describe the existing output formats supported by Scade
suite. Finally, in chapter~\ref{cha:possible-scenarios}, we define
possible scenarios based on the tools that can exploit these different
output formats and list the strengths and weaknesses of each approach.


\section{Overview of Scade}

Scade is a formal modeling language that provides both a graphical
(diagrammatic) and textual concrete syntax. Formal here means that the
intended meaning of a model can be defined unambiguously using a
mathematical framework. It is therefore possible to reason on models
and to prove properties on them with a very high-level of
confidence. More precisely, Scade models are synchronously clocked
state machines, interacting on infinite data streams, that can be
nested and intermixed with each other without limitations. Actually
the core of the Scade language is based on
Lustre~\cite{al91,halbwachs05}, a declarative, synchronous dataflow
programming language for reactive systems defined in the early 1980s
at the Verimag laboratory in Grenoble.

Another property of Scade is that it is an executable language with a
deterministic semantics, meaning that no randomness is involved in the
``execution'' of a Scade model; A model will thus always produce the
same outputs from a given set of inputs.  All these characteristics of
Scade (formal, executable and deterministic) made it possible to
develop the first certified code generators to C and ADA according to
the DO-178B and the EN50128. This is a benefit of using Scade with
respect to competing frameworks.

% We would like to find the same characteristics in the target language
% of our transformation.

\section{Motivations for the Adoption of Scade in the Project}

The Scade language is supported by a toolbox, the \emph{Scade suite},
that provides an integrated design and development environment. Scade
suite offers several services: simulation and debugging at the model
level; test case execution; model test coverage measurement; code
verification on models; \dots 

An extension of the tool, called \emph{Scade system}, provides
additional functionalities that are of interest in the context of the
openETCS project. Basically, Scade System provides systems modeling
and model generation based on the SysML standard and the Eclipse
Papyrus editor. In particular, Scade system offers ``partial'' gateway
from and to SysML and integration with Eclipse through the use of the
Eclipse Modeling Framework (EMF). This is relevant for the openETCS
project since Eclipse was selected as the de facto open source
platform for integrating the tools developed in the project, while
SysML was selected for modeling the architecture of the
system. 

Unfortunately, the SysML (architectural) model obtained from a Scade
specification does not preserve all the information. It only contains
information on the states machines and some information at the level
of the interfaces; the detailed code related to the functions and the
state transitions is left over. Therefore we cannot simply use the
output of the Scade system tool for our migration. Also, at the
moment, there is no proper way to import a ``Scade system'' model from
a ``Scade Suite'' model, even though such a gateway was requested by
the openETCS project more than a year ago. What is available at the
moment is a synchronization between models on ``shared'' blocks and
their inputs and outputs. It means that if you start using Scade
System to define the different software blocks, then you can refine
those blocks with Scade suite, evolve both models concurrently and
ensure synchronization of blocks with their data interfaces. There is
no support for models that have been only edited using the Scade
suite.

Scade has been used in production on several big projects that covers
many application domains: avionics, rail, automotive, \dots It allows
the production of rapid prototype as well as of safety related target
system software. The intrinsic qualities of Scade are enough to
motivate its choice for modeling the OBU functions. Nonetheless there
are also important contextual reasons. Most importantly, there does
not exists an open source solution with the same degree of
maturity. For example, even though the Scade editor requires a
moderately high learning curve, it is a solid piece of software that
has been tested on many real-size examples. Another circumstantial
reason for the adoption of Scade in the project is that some of the
partners already had a working knowledge of Scade. Also, many test
models related to the ETCS had been produced in the context of our
initial evaluation phase and were already available in the first
months of the project. Henceforth, considering the strong constraints
on the deadlines and the large amount of modeling work involved, Scade
was an obvious choice. In any case, it was certainly the choice
incurring the less amount of risk for reaching the objectives of the
project.


\section{Associated Drawbacks}
\label{sec:associated-drawbacks}

We can also list several drawbacks related to the choice of
Scade. First, the choice of a synchronous language implies that there
is a choice of implementation imposed on the users of the models
(namely the choice of a synchronous architecture). This restrict, or
at least complicate, some possible choices when implementing the
OBU. Indeed, the synchronous hypothesis is well-suited for control
functions that operate on a single computer since, in this case, it is
not difficult to have all the components share a common
clock. Nonetheless, this choice may have some non-trivial consequences
if a supplier decide to choose a more distributed architecture, where
different functions of the OBU are executed on different,
loosely-coupled hardware. This is exactly the kind of problems related
to the preservation of the semantics mentioned in the
introduction.

We can also mention drawbacks related to the choice of a closed-source
solution:
%%
\begin{itemize}
\item risk of vendor lock-in: actually this is at the core of our
  migration problem, since there are no easy way to export a Scade
  model to another modeling language. Note that, with the latest
  version of Scade suite, it is now possible to view Scade diagrams
  without buying the software (but of course it is not possible to
  edit them).

\item perennity of the models: there is a (albeit small) risk that the
  current Scade toolchain will be discontinued in the future,
  especially when we take into account the very long service life of
  railway equipments. We should also mention the problem of
  \emph{perennity of the model editor}. Indeed, we need to be able to
  continue editing models even when the tooling is discontinued.  This
  is a known limitation of closed-source solutions with respect to
  ``long term support'', since it is not possible to subcontract on
  the continuation of a tool when sources are not available.

\item difficulty to have special needs taken into account by the
  software editor, or services added to the baseline tools.
\end{itemize}



\section{Specificity of the OBU model}

The models are made available on the GitHub of the openETCS project
and a complete description is given in deliverable D3.5.3~\cite{d353}.
The specification of the OBU functions
(\url{https://github.com/openETCS/modeling/tree/master/model/Scade/System/ObuFunctions})
is a non-trivial example of system modeled with Scade; it has 13 main
sub-components each containing more than a dozen Scade diagrams. All
the facets of the Scade language are used but some characteristics of
the OBU models are worth mentioning:
%%
\begin{itemize}
\item The models use a very large number of (complicated) types that
  corresponds to the large number of variables and values defined in
  the ETCS specification. For instance there is a large number of
  ``datagrams'' defined in the sections related to the radio
  communication. Therefore we need to target a language with custom
  data type definition.

\item There is a heavy use of state machines in the specification, see
  for example the diagrams in the {\verb+ManageLevelsAndModes+}
  component\footnote{\url{https://github.com/openETCS/modeling/tree/master/model/Scade/System/ObuFunctions/ManageLevelsAndModes/Modes}}. This
  means that the OBU model is strongly hierarchical and therefore we
  need to target specification languages that are component-based
  or. at least, strongly modular.

\item The Scade models make use of most of the Scade v6 operators,
  like iterators on arrays or ``safe state automata'' for
  instance. Iterators are a recent extension to the Scade language
  that are not supported by other flavors of Lustre. In particular,
  this means that we cannot use a direct translation from Scade to
  Lustre for our needs.
\end{itemize}

\chapter{Overall Goal of the Transformation (What do we Want)}
\label{cha:over-goal-transf}


Solving our migration problem raises several issues, for instance
concerning the simplicity of the migration process or the coherence of
the resulting models. Concerning the first point, we are of course
interested by methods that are fully automatic and, if possible, at
least partially available (implemented). Based on the remarks provided
in the introduction of this report, we can try to establish a list of
the criteria that can be used to evaluate our migration scenarios.

%%
\begin{description}

\item[Open-Source Solution] we want to target an open-source solution,
  or at least a solution that fits in an open-source ecosystem of
  tools and languages. This is the core of the problems raised in this
  deliverable.

\item[Semantic Preservation] we want to loose as few information as
  possible during the migration (in the best case we aim at a
  transformation that is close to round-trip conversion). We also need
  a method for checking that the semantics of the models are preserved
  during the transformation.

\item[Simplicity] we need a solution that is already existing or at
  least partially completed. The target language should have all the
  available tooling: editor, simulator, formal verification, \dots As
  much as possible, the conversion should be automatic.

\item[Maintainability] migration raises also several questions related
  to the life cycle of the model, that is its ``evolvability'' in the
  future and its maintainability. Indeed a model is never a finished
  object, it is evolving and should be updated. Therefore we want a
  format that can be easily updated. This means that the structure of
  the resulting code and the meaning of the identifiers should be easy
  to understand.

\item[Code Generation] we need to have code generation tools, since
  the current Scade model is already used to generate code for
  prototyping.  The generated code should be easy to verify.

\item[Integration] we need an easy integration into the openETCS
  toolsuite; which means integration into Eclipse and, preferably,
  gateway to and from SysML through the Papyrus editor

\item[Certification] we need to take into account the issues related
  to the certification of the tools (since the openETCS project
  ultimately aims at EN50128 certification)

\item[Traceability] we need to preserve traceability information. At
  the moment, most of the traceability information contained in the
  Scade models are annotations in the comments that refer to relevant
  sections in the Subset-026 specification. In some parts of the model
  (e.g. \verb+Manage_TracksideInformation_Integration+ and
  \verb+calculateTrainPosition+), traceability is also supported using
  the SCADE requirements gateway.
\end{description}

%%%%%%%%%%%%%%%%%%%%%%%%%%%%%%%%%%%%%%%%%%%%%%%%%%%%%%%%%%%%%%%%%%%%%%

\chapter{Supported Output Formats from Scade}
\label{cha:supp-outp-form}

In this chapter, we study the different file formats available from
the Scade suite tool. Each format, both textual and graphical,
corresponds to a concrete syntax for Scade. This information is
interesting since, for each format, we can find tools that can import
Scade models that use this syntax; these tools give us possible
scenarios for exporting Scade models.  In the next chapter, we define
our migration scenarios as translations from one concrete syntax of
Scade to the concrete syntax of another language.

% Migration raises also several questions related to the life cycle of
% the model, that is its ``evolvability'' in the future and its
% maintainability. Indeed a model is never a finished object, it is
% evolving and should be updated. The most ambitious goal is to
% seamlessly replace the use of Scade suite by an open alternative. This
% will be the most interesting solution. 

\section{XSCADE}
\label{sec:xscade}

Xscade is the basic interchange format used in Scade suite. This is an
XML format keeping all editor-specific information; that is both the
information on the actual Scade code, but also the positions and the
properties of the graphical elements in a diagram (see Sect. 2 of the
Scade user manual, \verb+TechnicalManual_SC-TM-R16+, for a description
of this format)\footnote{unfortunately the Scade manuals are not
  freely distributable.}.

Since the last version of Scade suite, it is possible to view a
.xscade diagram without buying a license (a software token). Some
partial information on the diagrams can also be viewed from the Scade
System tool, which means that there should be an EMF description (an
\verb+ecore+ file) for at least a subpart of the Xscade format (namely
the part dealing with state machines). It is not possible to predict
whether ANSYS will be willing to freely distribute a separate Eclipse
plugin for viewing Scade diagrams or a separate Java model API for
manipulating the associated EMF models. Nonetheless Scade System
provides capabilities to develop specific export functions using OCL,
the TCL scripting language and a specific Java model API.

There exist a few open source tools able to import Xscade
files. Nonetheless all these solutions are based on ``retro-engineered
parsers'', therefore it is not clear if all aspects of the language
are supported and if these tools can withstand future evolutions of
Scade Suite. For example, we can find an Xscade importer on: (1) the
OSATE plugin, an Eclipse plugin for the AADL language; (2) the S3 tool
by Systerel\footnote{The S3 tool is not open source.}.

OSATE 2, see \url{http://www.aadl.info/}, is an Eclipse plugin that
provides an importer from XSCADE to AADL. Each component in a diagram
is mapped into an AADL system component; Scade component connections
are mapped using AADL event data port and state machines are
translated into AADL behavior state machine. We give some information
on S3 in the next chapter.

Solutions based on retro-engineered parsers are clearly subject to
risks if we want to use it concurrently with Scade modeling. This
approach is best suited to ``one shot'' scenarios in which Scade tools
are not used anymore once the models are exported. If not, we face the
risk of needing to maintain those parsers when the Scade tools
evolve.


\begin{center}
  \setlength{\fboxsep}{10pt}\fbox{\begin{minipage}{0.9\textwidth}
      \begin{description}
      \item[Advantage:] this is the most interesting
        format for conversion since no information is lost.
      
      \item[Disadvantage:] the format is complex and not very
        well-documented (the related XML Schema or ecore descriptions
        are not provided); no guarantee that retro-engineered parsers
        can be used on future versions of Scade.
    \end{description}
    \end{minipage}}
\end{center}

\section{Textual Scade format}
\label{sec:textual-scade-format}

SCADE Suite can also generate a \verb+.scade+ file from a diagram (by
default this file is named \verb+kcg_xml_filter_out.scade+) which
contains an equivalent of the XML Scade format into a structured
textual format. This format is also sometimes referred to as the
\verb+kcg+ format, since it is the input format used in Scade's code
generation tool. This textual format is very close to Lustre code and
it is quite low-level.

While the kcg format is very close to Lustre code, Lustre is not a
normalized language and the existing implementations have slightly
different syntax. For instance, Scade extends Lustre with new
operators, like iterators, that cannot be easily translated. Also,
during the serialization of Scade code, the names of ports and
variables came become quite obfuscated; therefore the resulting code
is quite verbose and difficult to edit. Also, the description of
nested state machines are flattened and some information on the
``hierarchical structure'' of the model are lost. Therefore a solution
directly based on this format would failed the criteria of
\emph{maintainability} listed in the previous chapter.

There are some works that have addressed the problem of converting
Scade's version of Lustre to other format. Most notably, Rockwell
Collins has developed a proprietary ``translation
framework''~\cite{whalen07} under a project sponsored in part by the
NASA Langley Research Center, that provides highly optimizing
translators from MATLAB Simulink and SCADE Suite to model-checking
tools and theorem provers. But this work is not distributed, even
commercially. It is also possible to find some open source tools that
provide a ``parser'' for .scade files, for example in the Scade2B
project (\url{https://github.com/cercles/scade2b}). There also exist a
lot of converter from Lustre to other formats that could be adapted to
conform to the Scade language.


\begin{center}
  \setlength{\fboxsep}{10pt}\fbox{\begin{minipage}{0.9\textwidth}
      \begin{description}
      \item[Advantage:] no information loss except for the graphical
        elements. This format is easier to parse than the Xscade XML
        format. There are many open-source Lustre implementation
        available (but none matching exactly the kcg format).
      
      \item[Disadvantage:] we loose information on the placement of
        graphical elements that may be needed if we target a graphical
        format; readability of the model is poor when generated from a
        Xscade file.
    \end{description}
    \end{minipage}}
\end{center}


\section{Executable Code (C or Ada)}

It is possible to obtain C or Ada code using the code generation
tools. As such, the resulting code is even less maintainable than the
Scade textual format. Nonetheless, we could imagine using part of this
output in another modeling language.

\begin{center}
  \setlength{\fboxsep}{10pt}\fbox{\begin{minipage}{0.9\textwidth}
      \begin{description}
      \item[Advantage:] many open source solution are available
      
      \item[Disadvantage:] difficult to update; too specific
    \end{description}
    \end{minipage}}
\end{center}

\section{SysML}
\label{sec:sysml}

The SCADE System tool provides support for developing Scade models
from a structural descriptions made of SysML Block Definition Diagram
(BDD) and Internal Block Diagram (IBD).  The full software
description---that is the precise behavioral description of each
block---can then be designed using SCADE Suite or in the form of
manually developed source code. Therefore it is possible to obtain an
architectural description of the OBU models in SysML, but the
resulting diagrams only contains information on the functions and
their interfaces; the detailed code related to the functions and the
state transitions are missing and should be added separately.

A possible solution to remedy this shortcoming could be to include the
``missing'' behavioral description inside SysML blocks, either as kcg
or as C code. To do this, it could be possible to use the scripting
capabilities of Scade Suite. Indeed, the SysML diagrams generated by
Scade system can be queried and transformed programmatically using
several methods: the Object Constraint Language (OCL), a query
language for EMF-based models standardized by the OMG; the TCL
scripting language; or a proprietary Java model API. We are not aware
of any previous research project where a comparable scenario has been
experimented with.

The gateway to SysML (and more generally to Eclipse and the Papyrus
editor) offers the best scenario for current integration with the
openETCS toolchain. The existence of these gateways is the reason why
Esterel Technologies often claim that the use if the Scade tool-sets
are not incompatible with open source projects.

\begin{center}
  \setlength{\fboxsep}{10pt}\fbox{\begin{minipage}{0.9\textwidth}
      \begin{description}
      \item[Advantage:] best possible integration with Eclipse;
        possibility to rely on the expertise of the CEA, which is a
        partner in the project.
      
      \item[Disadvantage:] the SysML gateway is fragile (susceptible
        to break or to change) and not as mature as the other Scade
        suite tools
    \end{description}
    \end{minipage}}
\end{center}


%%%%%%%%%%%%%%%%%%%%%%%%%%%%%%%%%%%%%%%%%%%%%%%%%%%%%%%%%%%%%%%%%%%%%%

\chapter{Possible Scenarios}
\label{cha:possible-scenarios}

We identify two main scenarios based on the different tools and
languages mentioned in chapter~\ref{cha:supp-outp-form}. In each case
we give a brief evaluation using the criteria defined in
chapter~\ref{cha:over-goal-transf}.

\section{Conversion to Lustre (and variants)}

Our first scenario relies on a transformation from the textual Scade
format to Lustre (see Sect.~\ref{sec:textual-scade-format})

The Scade language originates from (and still has some strong ties
with) the synchronous language Lustre. The Lustre language is still
being maintained by the Synchrone team at Verimag. The last update to
the Lustre V6 version is open source and has been released in 2014
(\url{http://www-verimag.imag.fr/Lustre-V6.html}). Open source tools
for Lustre includes a front-end compiler (\verb+lus2lic+) and a
back-end compiler to sequential, deterministic, C code.

Another open source implementation of the Lustre language can be found
in the \verb+prelude+ tool~\cite{forget:hal-00688490}, developed at
ONERA. Prelude is a high-level language built upon Synchronous
Languages (such as Lustre) and inherits their formal properties. The
language extends the syntax of Lustre and provides more high-level
constructs, in particular when defining multi-clocks systems. The
prelude compiler, called preludec, generates synchronized multi-task C
code, independent of the target OS, that can execute on either
mono-core or multi-core architectures.

As mentioned in Sect.~\ref{sec:textual-scade-format}, the main problem
with this approach is the maintainability of the models since the
generated code may be difficult to understand. Another problem is the
necessity to develop a transformation from the Scade format to one of
the academic Lustre syntax.

\begin{tabular}[c]{|p{3.5cm}|p{1.2cm}|p{8.8cm}|}

  \hline
  \multicolumn{3}{|c|}{Evaluation} \\
  \hline

  Semantic Preservation & GOOD & we deal with two formal languages that are very close to
                                 each other \\

  \hline

  Simplicity &  GOOD & \\

  \hline

  Maintainability & BAD & we can mitigate the problems with identifiers
                          by adapting the data dictionary but the absence of
                          notation for state machines in Lustre is a
                          big drawback; we loose the graphical
                          representation\\

  \hline

  Code Generation & GOOD &  even better options than with Scade\\

  \hline

  Integration & MIXED & no existing direct integration with Eclipse but no
                        obvious roadblocks \\

  \hline

  Certification & MIXED & no certified transformation\\
  
  \hline
  
  Traceability & MIXED & there are no specific support for
                         traceability in Lustre\\
  
  \hline

\end{tabular}


\section{Extension of SysML}
\label{sec:sysml-variant}

Our second scenario corresponds to the one described at the end of
Sect.~\ref{sec:sysml}, where we proposed to extend the SysML
(architectural) model with behavioral information extracted from Scade.
The idea is to add behavioral blocks to every function in the SysML
model that corresponds to the related Scade function.

There are several possible choice for the ``content'' of these blocks
(the language used for the behavioral specification): Lustre code; C
code obtained from code generation; fUML; HLL code; \dots

An obvious choice for adding behavioral block description to SysML
would be to rely on the ``Executable UML foundation'', fUML, that is
an OMG standard. fUML is an executable subset of UML that can be used
to define, in an operational style, the structural and behavioral
semantics of systems.

A more concrete choice is provided by HLL, the intermediate format
used in the S3 tool developed by
Systerel~\cite{S3,DBLP:conf/fmics/Petit-DocheBCFG15}. S3 is a formal
verification tool that provides a gateway from the Xscade format. The
language is typed and provides higher-order construction such as a
notation for state machines. Nonetheless some work would be needed in
order to better take into account (nested) state machines and to adapt
the language for its embedding inside a SysML block. The advantage of
this solution is that we could benefit from the expertise of Systerel,
that is a partner in the project, and that HLL has already been used
successfully on the openETCS Scade model, see for example the Systerel
VnV user story at
\url{https://github.com/openETCS/validation/blob/master/VnVUserStories/VnVUserStorySysterel/04-Results/e-Scade_S3/Scade_S3_VnV.pdf}


\begin{tabular}[c]{|p{3.5cm}|p{1.2cm}|p{8.8cm}|}

  \hline
  \multicolumn{3}{|c|}{Evaluation} \\
  \hline

  Semantic Preservation & MIXED & we need to choose a ``semantics''
                                  for SysML (since different tools may
                                  have different interpretations of the language, like
                                  for examples transitions in State Diagram)\\

  \hline

  Simplicity &  MIXED & While there are plenty of support for SysML,
                        there are no tools supporting an ad-hoc extension
                        with behavioral blocks\\

  \hline

  Maintainability & MIXED & need to be evaluated (but seems better
                            than with a pure Lustre solution) \\

  \hline

  Code Generation & BAD &  no existing solution off-the-shelf\\

  \hline

  Integration & MIXED & no existing direct integration with Eclipse but no
                        obvious roadblocks \\

  \hline

  Certification & BAD & no existing solution\\
  
  \hline
  
  Traceability & GOOD & with this solution we  have very strong links
                        between the architectural and detailed design
                        models. Possibility to use the Eclipse
                        traceability standards all the way.\\
  
  \hline

\end{tabular}





%%%%%%%%%%%%%%%%%%%%%%%%%%%%%%%%%%%%%%%%%%%%%%%%%%%%%%%%%%%%%%%%%%%%%%

\chapter{Conclusion}


It should not come as a surprise that there is no existing solution
that satisfies all of our needs. There are some reasons for this
situation that goes beyond the simplistic argument of vendor lock-in:
first, there is a relatively small user-base for Scade (and therefore
less incentive for third-parties to develop export tools); next, there
is a lack of compelling scenarios for exporting Scade models to
another format, except for code generation. Indeed, Scade targets
mainly the detailed design phase (the latter design step of the
traditional development cycle) and therefore Scade models have no use
after the code generation or model verification activities, which are
both supposed to be performed using the Scade tools.

In this report, we propose two plausible scenarios for migrating away
from Scade and give some references on projects that already provide
import methods from Scade. Unfortunately none of these projects
provide a solution that could be used directly and some software
development is therefore needed.

Another possible scenario---that is not reported in this deliverable
but that was temporarily evaluated---is to consider a migration to a
hybrid modeling language, such as Simulink. Actually, considering our
goals, a more appropriate target would be Scicos/Xcos
(\url{http://www.scicos.org/}), that provides an open-source
alternative. Simulink is a de facto standard when working on
heterogeneous models, especially from the point of view of the control
and command engineers. There are some reasons against a shift to a
Simulink-like formalism. For instance, Simulink or Scicos are mainly
used to model and simulate systems, but rarely used to implement
critical software\footnote{In the domain of high-integrity software,
  even if initial models of the system with its plant are often
  described with Simulink in the initial phases, software is more
  often reimplemented in Scade or imperative code.}. This is mainly
related to the fact that it is very difficult to ensure the
consistency of the models with respect to the run-time behavior of the
generated executable code.  These problems could be mitigated by
restricting to a formally defined subset of Scicos. Nonetheless, this
solution was ultimately put aside since there are no tools available
for exporting to Simulink or its open-source. Also, and more
importantly, this migration scenario was the one with the more risks
associated to it and the one that required the most implementation
efforts.


% it is also necessary to acknowledge the fact that, exactly like
% software engineering, \emph{model engineering} is essentially a social
% process and we cannot expect that a transition will not disrupt the
% work of the modeling team and the quality of its output.


%%%%%%%%%%%%%%%%%%%%%%%%%%%%%%%%%%%%%%%%%%%%%%%%%%%%%%%%%%%%%%%%%%%%%%

\bibliographystyle{unsrt}
\bibliography{ref}


%===================================================
%Do NOT change anything below this line

\end{document}


%%% Local Variables:
%%% mode: latex
%%% TeX-master: t
%%% End:
